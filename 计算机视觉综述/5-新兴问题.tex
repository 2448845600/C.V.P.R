\section{计算机视觉新问题}
% 解决了哪些问题,还有哪些没解决

\subsection{3D-version}
前文介绍的计算机视觉算法都是基于2D场景,但是实际场景中,图片包含3D信息,利用额外的3D信息更准确的得到类别和位置情况。在对3D信息的利用上,有许多方法:
\begin{itemize}
\item 基于体素的卷积神经网络。从RGB-D图中得到体素信息,设计专门的神经网络处理体素。
\item 基于多视角的卷积神经网络。利用点云图分类。
\item 基于特征的深度神经网络。将3D数据转换为多维矩阵,送入神经网络。
\end{itemize}

\subsection{视觉安全问题}
随着深度神经网络的发展,安全问题浮出水面。部分研究人员开始在图片中增加噪声、干扰,达到“欺骗”神经网络的目的。由于深度神经网络在安防、身份识别等领域的广泛应用,安全问题迫在眉睫。

\subsection{强化学习在视觉中的应用}
生成对抗网络(GAN)在各类问题上均有使用,在计算机视觉领域\cite{mao2017least,mirza2014conditional,berthelot2017began,bousmalis2017unsupervised,isola2017image}也不例外。在数据生成、图片与文本结合领域有一定的应用。