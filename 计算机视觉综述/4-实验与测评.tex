\section{实验与测评}
% 定性测评(主管评测),定量评价(正确率等指标)

深度神经网络的发展一直是实验创新先于理论突破。本章节总结实验与测评的相关技巧。

\subsection{实验}

\subsubsection{数据集}
COCO,ImageNet是现阶段最大最全的两个数据集,在图像识别、目标检测、语义分割领域广泛使用,是衡量算法性能的重要工具。不同的计算机视觉任务还有很多对应的数据集,如车辆识别、行人识别、艺术品等。

\subsubsection{训练}
\paragraph{end-to-end}
深度神经网络在训练上的追求。一端输入数据,一段出结果,训练过程不需要额外干预。

\subsection{测评}
在算法、模型的评估中,TP、FP、FN、TN是最基础的概念,如表\ref{tab:table1}。
\begin{table}[!htbp]  
\centering
\caption{样本分类}
\label{tab:table1}%添加标题 设置标签  
\begin{tabular}{ccc}  
\hline  
 & 正类 & 负类\\
\hline  
分类   & TP & FP\\
未分类 & FN & TN\\
\hline  
\end{tabular}  
\end{table}

在理解这些定义后,\textbf{准确率}(accuracy),\textbf{召回率}(recall),\textbf{精确率}(precision)就可以公式化表示。

\begin{align*}
Accuracy &= {\frac{TP+TN}{TP+FP+TN+FN}} \\
Precision &= {\frac{TP}{TP+FN}} \\
Recall &= {\frac{TP}{TP+FP}} 
\end{align*}

\paragraph{mAP}定义较为复杂,先介绍P-R图:制定一个类别,根据学习器下每个样本对这个类的预测结果进行排序,排在前面的是学习器认为“最可能”的正例样本,最后的是“最不可能的”.按照此顺序逐个把样本作为正例进行预测(1个正例,2个正例,…),则每次可以计算出当前的R值和P值,然后以R值为横轴、P值为纵轴记性作图.就得到了P-R图. 

通过P-R图可以定义AP(average precision),就是这个曲线下的面积,这里average,等于是对recall取平均。而mean average precision的mean,是对所有类别取平均(每一个类当做一次二分类任务)。现在的图像分类论文基本都是用mAP作为标准。使用AP会比accuracy要合理。对于accuracy,如果有9个负样本和一个正样本,那么即使分类器不做推测,直接全部判定为负样本,其accuracy也可以达到90\%。但是对于AP,recall=1处precision会低至0.1,曲线下的面积就会反映出性能。连续图中AP应该是P对R在[0,1]上的积分,实际上P-R图中的值是离散值,化为sum即可。而mAP是对所有类别的AP值的平均,也就是所有的AP值求和除以类数。