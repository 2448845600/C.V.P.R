\section{前言}
%背景和动机描述
\subsection{背景}
计算机视觉在生产生活中应用广泛,如无人车辆需要分析摄像头拍摄的图片以判断路况\cite{sheth2018design, cardarelli2017cooperative};安防系统比对图片数据库锁定嫌疑人\cite{li2015convolutional,yang2016wider,jiang2017face,farfade2015multi};体感游戏对用户姿态分析实现人机交互\cite{belagiannis2017recurrent,lifshitz2016human}。这些需求要求更加迅速、准确、可靠的计算机视觉算法。传统的计算机视觉算法一般通过特征工程、滤波器、机器学习等方法\cite{zuo2018learning,jordan2015machine,lison2015introduction,meyer2015support},取得了一定的成果,但是发展陷入了瓶颈,到大规模应用还有很大差距。近些年兴起的深度学习在计算机视觉领域掀起轩然大波,占据各个视觉竞赛榜首\footnote{http://cocodataset.org}\footnote{http://www.image-net.org/}\footnote{https://www.kaggle.com/},相关算法层出不穷,论文数量与日俱增,性能逐步提高。

但是我们也看到,深度学习在部分计算机视觉问题上研究不充分,如三维计算机视觉\cite{song2015sun,song2016deep,whelan2015real}、复杂条件下的识别\cite{wu2016robust}等;除此之外,深度学习的发展也带来了意想不到的问题,如视觉攻防、信息安全\cite{whelan2015real,rao2015computer}等;深度学习在其他领域的发展也为计算机视觉带来新思路,如生成对抗网络(GAN)\cite{faubel2016cilia,goodfellow2014generative}在视觉领域的应用。这些新领域新问题新思路有助于得到更加全面高效安全的计算机视觉算法。

\subsection{动机}
计算机视觉领域发展迅猛,每年的会议期刊应接不暇,顶会文章就有数千篇。对于初学者,如何在这些年万余篇论文中把握住计算机视觉的发展方向是个难题。本人也是该领域的初学者,希望通过整理最近阅读的论文、博客、代码等学习资料,浅显的归纳总结计算机视觉的成熟经验,对当下热点做一些前瞻性预测,帮助自己和同行更好的进行研究。
